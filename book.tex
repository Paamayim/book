\documentclass[12pt]{book}

\usepackage[margin=1in]{geometry}
\usepackage{amssymb, amsmath, amsfonts}
\usepackage{titlecaps}
\usepackage{xargs}
\usepackage[colorinlistoftodos,prependcaption,textsize=tiny]{todonotes}
\usepackage{tikz-cd}
\usepackage{graphicx}

\newcommand{\cdr}[1]{\arrow[swap]{r}{#1}}
\newcommand{\cdl}[1]{\arrow{l}{#1}}
\newcommand{\cdd}[1]{\arrow{d}{#1}}
\newcommand{\cdu}[1]{\arrow{u}{#1}}

\title{Ideas Unthinkable}
\author{Sandy Maguire}
\date{\today}

\newcommand{\difficult}{}
\newcommand{\subtitle}[1]{}
\newcommandx{\unsure}[2][1=]{\todo[linecolor=red,backgroundcolor=red!25,bordercolor=red,#1]{#2}}
\newcommandx{\change}[2][1=]{\todo[linecolor=blue,backgroundcolor=blue!25,bordercolor=blue,#1]{#2}}
\newcommandx{\info}[2][1=]{\todo[linecolor=green,backgroundcolor=green!25,bordercolor=green,#1]{#2}}
\newcommandx{\improvement}[2][1=]{\todo[linecolor=purple,backgroundcolor=purple!25,bordercolor=purple,#1]{#2}}
\newcommandx{\missing}[2][1=]{\todo[linecolor=purple,backgroundcolor=yellow!25,bordercolor=yellow,#1]{#2}}
\newcommandx{\thiswillnotshow}[2][1=]{\todo[disable,#1]{#2}}
\newcommand{\aside}[1]{#1}
\renewcommand{\value}[1]{\ensuremath{\text{\textit{#1}}}}
\newcommand{\type}[1]{\ensuremath{\text{\texttt{#1}}}}
\newcommand{\tuptype}[1]{\type{(#1)}}
\newcommand{\func}[1]{\ensuremath{\text{\texttt{#1}}}}
\newcommand{\typeof}{\ensuremath{\text{ \texttt{::} }}}
\newcommand{\what}{}
\newcommand{\defn}[2]{\renewcommand{\what}{#1}\textbf{Definition "\titlecap{#1}":} #2}
\newcommand{\exprob}[2]{\textbf{Problem:} #1 \\ \rule{\textwidth}{1pt} \\ \textbf{Solution:} #2 \\}
\newcommand{\lesson}[1]{\textbf{Lesson:} #1}
\renewcommand{\cite}[1]{}
\newcommand{\coord}[1]{(#1)}
\newcommand{\fig}[3]{
    \begin{center}
    \includegraphics[scale=#3]{#1} \\
    #2 \\
    \end{center}
%    \label{fig:awesome_image}
}

\includeonly{}


\begin{document} \maketitle

\chapter{Introduction}
Iktsuarpok is an Inuit word which roughly translates in English to "the frustration of waiting for someone to turn up."
Think about that feeling; it's something you're almost certainly familiar with, but, I'd suspect, never in the sense of
iktsuarpok. You probably categorized the feeling as "frustration" or "irritability", both of which are close, but
neither of which is as accurate as iktusarpok. English doesn't really have a word for this feeling, but somehow, we
manage to get by. Our existing words do a adequate-enough job of expressing our emotions, and so we don't see the need
to spend the effort finding and spreading a new word just for this purpose.

The Sapir-Whorf hypothesis\improvement{Be more science-y here}(or linguistic relativism), a theory widely considered
to be true by scholars of linguistics, states that the language you speak influences the thoughts you can have. For a
particularly illustrating example of this hypothesis, quickly think to yourself which direction is south-west relative
to you right this instant.  Unless you have recently looked at a map, your orienteering isn't likely to be very
accurate. Humans, it would seem, do not have very good internal compasses. However, this claim turns out to have very
few legs to stand on indeed; the Australian Aboriginal people the Guugu Yimithirr
\cite{http://en.wikipedia.org/wiki/Relative_direction} have no issue with this task whatsoever, and the reason why is
quite illuminating: they do not have words for "left" or "right". When asking a Guugu Yimithirr which hand he writes
with, he might respond with "my Eastern-most hand." The fact that their language lacks words for relative directions
means that the Guugu Yimithirr need an alternative means of describing direction, and so they instead keep track
mentally of the cardinal directions to provide the same purpose.

Isn't it quite remarkable to realize that, of all of the \unsure{is this number accurate?}millions of words in the
English language, there still exist ideas that are difficult to express? Ideas that would take so long to describe that
by the time you got to the end of your sentence, people would have forgotten the beginning? Linguistic relativism
implies that this indeed must be the case; that there are thoughts English is poorly adapted to handle.

\improvement{I don't like the way this is worded.}That's What this book is for; to teach you a language capable of
expressing the ideas you never even knew were inexpressible. As it turns out, these ideas are \textit{profoundly
powerful}; they are the reason that engineers and computer programmers are so highly paid. Unfortunately, as it is with
all things, we must learn to walk before we can run. It is an inescapable consequence of this that a lot of the ideas
will seem obvious at first, but that is only because you have had decades of practice with your current thinking
patterns.

I liken the experience of this as similar to learning to ride a bicycle\todo{maybe a car analogy would make more sense}.
When you first start, you're not any good at it, and every time you get on there is a non-zero chance that you will end
up hurting yourself. For all intents and purposes, early on, it's easier to get where you're going by walking than by
riding your bicycle. However, once you've gotten familiar with the bike, and have learned to not fall over, even in the
absence of your training wheels, you're more mobile than ever before. I think this is a fair metaphor for what we're
trying to do here. Give this book the benefit of the doubt; just because your current system works better at first,
doesn't necessarily it will be better in the long run.\improvement{this sounds patronizing}

\section{Things I want to say that I stole from that stats book}
We have tried to write this book in an intuitive fashion, emphasizing concepts rather than mathematical details.
The symbol \difficult indicates a technically difficult section, one that can be skipped without interrupting the flow
of the discussion.



\chapter{Recursion}
A central theme of this book is one of \textit{structure}. By that, we mean that a primary interest is describing
the structure of \textit{things} in a systematic way. One of the most elegant patterns of structure is that of
recursion: things described in terms of themselves. Most of the concepts in this book can be described thusly, so it
seems fitting that we study recursion itself as motivating introduction both to the \textit{kinds of things} this book
will be exploring, and furthermore as a strong foundation itself.\improvement{not a beautiful sentence}

\section{Introduction}
\fig{recursion/monalisa.jpg}{A picture of the Mona Lisa holding the same picture.}{0.75}
My first encounter with the idea of recursion was on the cover of comic book my father had lying around our library when
I was a boy. The cover featured the main character reading a copy of the comic book that \textit{I} was reading. Clearly
visible on the cover of his book was \textit{him reading the same book}. My eight-year-old mind wasn't entirely capable
of comprehending what was happening, and I spent a significant amount of my childhood wondering about how such a thing
could possibly make exist. Of course, the book could exist because the cover in the cover of the cover was so small that
the human eye couldn't possibly make it out, so the artist could stop drawing, his job being complete.\unsure{maybe
describe the mona lisa instead?}

This fact, of course, is merely one of pragmatics. With a powerful enough imagination, you realize that there is nothing
\textit{in principle} which stops progressively smaller copies of my comic book protagonist from reading his own book,
in a series which \textit{never ever stops}.




\chapter{Information}
\chapter{Types}
\section{Motivation}
\missing{mention how the point of this is to accurately constrain what we're talking about}
At its core, computation\info{and all of the things that go along with it} really nothing more than a series of
transformations from one thing to another, and from the result of that transformation to something further along. This
is pretty easy to observe in the real world; every firm involved in manufacturing chain is a good example. The mining
corporations transform the raw earth into crude materials, which are then taken by the chemical engineers and
transformed into refined materials. These refined materials are sold to factories, which transform them into assembled
goods, and will usually sell them to other factories to make other assembled goods. Each individual organization
provides one individual piece of the overall manufacturing transformation -- the whole of which is made up of nothing
but the individual pieces. The structure of a transformation is recursive: it is made up of smaller transformations, and
makes up larger ones.\improvement{Drive in the point that this goes further in both directions: eg. every person at an
organization is responsible to transform their pay into something valuable for the company.}

In fact, one might be tempted to view ones own life as a transformation, from what they were born with to what they want
to do with it. Though human-level problems are unfortunately all-too-often too complicated for our
mathematical\unsure{do we want to use this word?} tools, taking this view is likely apt to help people find their
way.\change{blegh. not sure this is thematically appropriate}

However, in order to study large transformations of any value whatsoever, we must first present ourselves with toy
problems and analyze how they work, trusting that our discoveries will be useful when dealing with more complicated
examples. It is hard to perform a transformation if you are uncertain about what it is you're working with. The study of
types is thus the necessary bookkeeping of \textit{what we're dealing with} and \textit{what we can do with it}. The
reader is encouraged to thoroughly understand (and ideally, have internalized) this chapter before proceeding, as the
remainder of the book builds heavily upon it.


\section{Types and Values}
The study of types is motivated as a necessary component of studying \textit{values}, and as such, it would seem to make
sense to first consider values.

\defn{value}{A \what is a particular noun, i.e. it's a thing that exists.}

\aside{Such a definition seems unnecessarily vague, but this is in fact intentional, and, as we will see, rather
characteristic of most of the definitions in this book. The concepts we are trying to learn words for are purposefully
vague; we are attempting to build mental scaffolding that works regardless of what in question we are thinking about.}

For the time being, we will consider values to be things that exist in the real world, things you can physically
experience. For example, my copy of \value{Crime and Punishment} is a value. So is \value{George W. Bush}, and the color
\value{brown}. The word \value{"Charlie"} is a value, and so is my friend \value{Charlie} -- but these two values are
distinct from one another, for reasons that we will see shortly.

\defn{type}{A \what is an arbitrary category, grouping several values together by functional purpose.}

Types are no more difficult of a concept to grasp than values were. A type is simply a classification which describes
several different values. For example, \type{Color} is a type with values \value{brown}, \value{green},
\value{orange}, along with a slew of others whose enumeration is left as an exercise to the reader. \value{Cola} and
\value{grape soda} are both values of type \type{Soft Drink}, but as a subtle point, they are also values of type
\type{Beverage} and also \type{Liquid}, because all soft drinks are beverages, and all beverages are liquids. A
value can belong to many types.\change{make this not be an example of polymorphism}

It is important to note that types don't \textit{really} exist in reality -- they exist only in our minds. While you can
point to a specific grape, there is no magical property of "grapeness" floating out there in the universe. Fortunately,
most of us manage to live our everyday lives without belaboring this philosophical point as we merrily talk in sweeping
generalizations about "how to get ahead in your career" or the implications of a politician's economic
policy.\improvement{come up with better examples here} A type is perhaps best thought of as distilling a concept down to
its simplest form under which its properties are salient for the discussion at hand; if we are desperately looking for a
place to sleep we will happily accept any value of type \type{Bed}, which distills the underlying piece of furniture
into its most salient property: being sleepable-upon.

As we saw earlier, as in the soft drink example, a value can belong to many types. When these types happen to be
specific cases of more general types (for example, \value{cola} is of types \type{Soft Drink}, \type{Beverage} and
\type{Liquid}; but \textit{all} beverages are liquids, by definition), we call this "polymorphism", literally:
being of many-form\unsure{is this true?}.

\defn{polymorphism}{A value is said to be polymorphic to all of the types it's type is by definition.\change{this is a
shitty definition that isn't going to help anyone}}\missing{explicitly describe how we can replace any type with
something which is polymorphic to it}

That's a little wordy, so really what this means is that, because \value{Socrates} is of type \type{Man}, and all
\type{Man} are of type \type{Mortal} by definition, \value{Socrates} is also of type \type{Mortal}. However, the
converse is not true: which is to say, not all values of type \type{Mortal} are of type \type{Man} (for example,
\value{Mr. Ed}, the talking horse).

The attentive reader will have noticed that if types can be polymorphic to other types, there is no solid ground we can
stand on to state that \value{cola} should be an object of type \type{Soft Drink}, but \value{soft drink} should not
be of type \type{Beverage}. Indeed, there is no such distinction between what should be considered a value and what a
type, as the procedure for assigning types to values is purely arbitrary.\improvement{this doesn't seem very clear} As
such, we shall not bother ourselves too much with the philosophical issue of where the line between a value and a type
should be drawn. We shall instead follow the rule of "the individual things we care about are values, while the
collection of them as a whole is a type".

An interesting consequence of this realization is that types themselves can have types, that is to say, there exists a
type \type{Type}, whose values are types themselves. An example of a value of type \type{Type} would be
\type{Color}. This is not to say that any individual color is a type, but that the \textit{idea} of colors is itself a
type. This kind of recursion certainly takes some getting used to, and is in fact the actual reason why we are
developing type theory: so as to ensure we don't get confused when we are talking about several different layers of
"meta" simultaneously.\footnote{It is the author's humble opinion that confusion of this sort is the primary reason why
contemporary philosophy doesn't seem to get very far.}

There is a conceptual shift here whose internalization is necessary before we can continue. The structure of values and
types is recursive, and akin to the idea that the relationship between houses and cities is the same as the relationship
between cities and states, which itself is the same as the idea between states and countries. The relationship is the
same across all levels, namely that the larger of the two is made up of a collection of the smaller.\change{word this
better plox}

Assuming the reader has followed us so far, it seems prudent to now mention a few interesting types. The first of which
is \type{Type}, which we have already discussed. Another is \type{Any}, the type to which all values are
polymorphic (i.e. absolutely everything is of type \type{Any}). The corollary to \type{Any} is, of course,
\type{None}, which has no values. The reason for this will be explored later.\change{it might make sense not to
introduce this until we need it.}


\section{Functions}
If values are the nouns of our newly formed language, functions are correspondingly the verbs. Functions are
transformations from one type to another.

\defn{function}{A \what is a transformation. It changes objects of one type into another.}

An intuitive example is the function \func{buy}, which transforms \type{Money} into \type{Purchasable} (i.e.
something which can be bought). Note that we are not talking about an individual transaction here; we are discussing the
idea itself of purchasing. Symbolically, we will represent this function as:

$$ \func{buy} \typeof \type{Money} \to \type{Purchasable} $$

More generally, however, a function is always of the form:\improvement{it would be nice to have more examples here}

$$ \func{function} \typeof \type{Type} \to \type{Type} $$

We will call the type on the left of the arrow the "input" or "parameter" of the function, and the type on the right the
"output" or "return type".

And just when you thought these ideas couldn't get any more twisty than they already are, because we're discussing type
theory, functions are not just transformations between types, they \textit{themselves} have types. Specifically, the
function \func{buy} above is of type $\type{Money} \to \type{Purchasable}$. Types of this nature, we shall call
"function types". You will notice the type of this function is just what we wrote after the \typeof symbol above.
Indeed, for convenience, we shall henceforth use this symbol to mean "of type" in the remainder of this
book.\improvement{combine this paragraph with the next.}

Remember, all types are polymorphic to the type \type{Type} (which, we are now able to write as $\type{Type} \to
\type{Type} \typeof \type{Type}$), and this includes function types themselves. By the rules of what it means to be
polymorphic, there is nothing stopping us from writing functions of the form:

$$ \func{hypothetical} \typeof \type{Input1} \to \type{Input2} \to \type{Output}$$
\missing{we need to describe precedence and why sometimes you need parens.}

For example, creating a great artistic masterpiece might be a function of type $\type{Time} \to \type{Energy} \to
\type{Creativity} \to \type{Art}$. The right-most type in a chain of function arrows of this sort is always considered
to be the output of the function, and with all others being inputs. In this case, creating great art requires time,
energy and creativity.

Functions are perhaps the most crucial subject\unsure{is this a substantiated claim?} covered in this
book.\missing{discuss how just coming away with a strong understanding of this is enough to warrant considering the book
a success}


\section{Variable Types}
As powerful as our type system already is, it quickly falls apart in some very common domains. As an illuminating
example, let us consider how we might express the idea of a list in our language of types. For our purposes, we will
define a list as a collection of zero-or-more things. A phone-book might be a list of \type{Name}s (with a size perhaps
in the millions), while a shopping list could be a list of \type{purchasable}s (with a significantly smaller size). For
simplicity, we will assume the only thing we might want to do to a list is to add an item to it, with a function
$\func{addToList}$.

It is clear that there should be some similarity between lists containing different types. In fact, the underlying type
of the list is almost entirely unrelated to anything we might want to do with the list, with one key exception: adding
items to the list.\change{make addToList closer to the next paragraph}

How might we go about determining the type of \func{addToList}? There are two relatively obvious choices, one of which
is to ignore the contents of the list entirely, substituting in our special type \type{Any}:

$$ \func{addToList} \typeof \type{Any} \to \type{List} \to \type{List} $$

or, alternatively, we could create multiple functions of the form:

$$ \func{addToNameList} \typeof \type{Name} \to \type{NmaeList} \to \type{NameList} $$

one for each type of list we might ever want.

Unfortunately, neither of these solutions is very good. Using \type{Any} means we can create all of our list
machinery once and only once, at the expense of losing the safety of our typing mechanism\todo{define this}. Due to the
fact that \func{addToList} takes a parameter of type \type{Any}, some unscrupulous character might try to add a
\type{Banana} to our phone book -- clearly nonsensical, but certainly allowed by the type system. Remember, the point of
studying types is to ensure we don't do anything nonsensical.\missing{oh yeah, mention this earlier too! -- probably in
motivation?}

The solution to this problem is actually a combination of both methods. Instead of creating a variant of
\func{addToList} for each type we might ever want, instead we will describe a \textit{rule for how to create such
functions}. This has the benefit of only needing to be written once, but retaining type safety ensuring that we don't
shoot ourselves in the foot. We thus redeclare \func{addToList} as follows:

$$ \func{addToList} \typeof \type{A} \to \type{List[A]} \to \type{List[A]} $$

where \type{A} is considered to be a value of type \type{Type}. We can consider this to be a rule to generate unique
functions \func{addToList} for every possible type \type{A} imaginable by playing "mad libs" -- we will replace \type{A}
with the type we care about. Our phone book from above can thus be considered of type \type{List[Name]}, while a bunch
of bananas would be a \type{List[Banana]}, and it now being typedly-impossible to add a \type{Banana} to a
\type{List[Name]}. We have successfully avoided shooting ourselves in the foot.

We call such a construction to be a "variadic type", and it allows us to describe "containers" of types: additional
structure on top of types which doesn't depend on the type it surrounds.

\missing{talk about multiple parameters, and how to imply covariance}

\defn{variadic type}{A rule for creating types by substituting in types.\change{completely shit definition}}

\exprob{What might a function $\type{Number} \to \type{Number} \to \type{Number}$ do?}{We are given two \type{Number}s
and asked to make a third out of them. The possibilities of what this operation might be are endless, but without any
further information, adding or multiplying the two is a good bet.}
\improvement{make the distinction between functions and signatures earlier.}
\lesson{Functions are not uniquely categorized by their signatures. This is why we say function signatures are a type --
there are lots of functions that can satisfy a signature.}


\section{Pairs and Tuples}
\unsure{maybe we should use cartesian product notation for tuples instead}
Finally, we find ourselves at the last type we'll need to discuss for the purposes of this book, and it's the secret to
building interesting types which consist of multiple pieces. The type in question is called \type{Pair[A,B]}--a variadic
type with two parameters, \type{A} and \type{B}. Note that we are free to assign \type{A} and \type{B} independently of
one another, and because of this, \type{A} and \type{B} \textit{can} be the same type, but they don't need to be. In
fact, if we wanted to, we could define only one of these parameters, for example, $\type{NamedObject[A]} =
\type{Pair[A,Name]}$ might be a pair representing something and its name.\todo{talk about equality of types as a matter
of convenience} We might be able to describe a person's name and age by $\type{Person} = \type{Pair[Name,Number]} =
\type{NamedObject[Number]}$.

If it's potentially useful to be able to pair \textit{two} different types together, you might wonder whether there
exists a type \type{Triplet[A,B,C]}? Indeed there does, but it's simply an alias for $\type{Pair[A,Pair[B,C]]}$, in
other words, it is a pair attached to an object. We will generalize this phenomenon to as many types as we want to
attach together, by connecting pairs together, always as the \type{B} type in the signature of \type{Pair[A,B]}. For
convenience, we will call this procedure "tupling", and the result to be a "tuple" (the suffix of words like quadruple,
quintuple, sextuple, etc.). In an attempt to shorten the amount of typing (no pun intended)\unsure{do I want to keep
this here?} necessary, we will write a tuple as $\tuptype{A,B,C,\dots} = \type{Pair[A,Pair[B,Pair[C,\dots]]}$. In the
special case where all the types inside of the tuple are the same, we shall describe it as $\type{A}^n$, where $n$ is
the number of values we want. For example, our earlier\missing{I didn't actually use this example} type denoting a
person's name as a \type{(Name, Name)} could more tersely be described as $\type{Name}^2$.

Remember, tuples are just a means of convenience. If we really wanted to, we could describe everything as a chain of
\type{Pair[A,B]}s, but it would be a lot of work and even the most fastidious student would likely get tired and give up
before ever getting anywhere.


\section{Further Thinking about Types}
As mentioned at the beginning of this chapter, having a strong understanding of types (what they are, how they are
constructed, and how to reason about them once you have one) is crucial to success in the rest of the book. With that in
mind, here are a few practice problems to help evaluate your understanding.

\improvement{add more of these}
\unsure{it might make sense to inline these as we go, to give people a sense of the power we are developing}
\exprob{What might a function $\type{List[A]} \to \type{Number}$ describe?}
{What kind of operations might transform a list into a number? If it were a list of numbers, our answer might be more
complicated, but since it's a list of arbitrary type, the only seemingly meaningful answer is to count the number of
items in the list.}

\exprob{What might a function $\type{List[A]} \to (\type{A} \to \type{Boolean}) \to \type{List[A]}$ do?}
{The key to solving these seemingly arcane incantantations is always to try to break it down into smaller, more
    manageable chunks. First we must consider our inputs: we are given a \type{List[A]} and a function $\type{A} \to
    \type{Boolean}$, a function which turns an element in the a list into a yes or a no. Since our output is also a
    \type{List[A]}, whose structure we known nothing about, except that it is a list, it implies we must be operating on
    the list itself. A strong candidate for what this function does is filtering--determining whether or not to keep
    each item in a list.}

Since we are able to come up with relatively-intuitive meanings behind arbitrary function signatures, it implies that
this type system is doing some work for us. Somehow it is capable of expressing ideas we can think in English in a more
structural way. Imagine that every word in English were designed by putting together fragments of meaning. In such a
system, you would know what a word meant even if you had never heard it before, because the \textit{structure} of the
word would tell you what it meant.

As one extreme example, consider the word "pneumonoultrascopicsilicovolcanoconiosis"\footnote{this is a real word that
I've known since it was the last word in my grade 5 spelling bee.} By decomposing it into its parts \textbf{pneu + mono
+ ultra + scopic + silico + volcano + coni + osis}, we can get an idea of what it might mean: \textbf{air + single +
extreme + small + rock + volcano + disease} or "getting sick from having small pieces of volcanic dust in your lungs".
As a matter of fact, most of the non-everyday English words have this kind of structure if you know enough Greek or
Latin to be on the lookout for it.

However, English does \textit{not} have the property of being able to build words whose meaning you know--going
backwards from what we did in the volcanic dust disease example.\todo{mention hoogle}\unsure{is my point ruined by
saying we don't need a dictionary anymore?}



\chapter{Spaces}
\subtitle{The Language of Possibilities}
\subtitle{The Language of Low Structure}
\section{Motivation}
Types give us a strong language to ensure we're talking about the things we want to be talking about. They stop us from
doing stupid and nonsensical things, like comparing \type{Human}s and \type{Apple}s. However, from the point of view
of the type system, all people are created equal--they are all completely interchangeable with one another. While this
might be nice from an idealistic perspective, our personal experience describes all too well how your boss and
significant other are \textit{very different} from one another, despite the fact that they are both of type
\type{Human}. While types give us the ability to distinguish between different kinds of things, the language of spaces
is that of the differences between members of the same type. While types tell us how things are the same, spaces tell us
\textit{how things are different}.\change{this is a little rambling}

Having a strong grasp of what spaces are (and how they can be manipulated) will bestow upon you the ability to
identify specific values of a given type, find common features between them which is invisible to the type system, and
consider possibilities that you have never seen before.


\section{A Gentle Introduction}
Consider a map of the Earth. Given a latitude and longitude, you can accurately describe any location on the planet. You
might not be marveling now, but humanity didn't figure out coordinate systems until around the 17th
century\footnote{Although there is some evidence of the ancient Greeks using something roughly isomorphic to coordinate
systems.}--several centuries after\unsure{write something here to tie that into context}. Latitude and longitude are
just numbers, relative to a starting point defined only by fiat (the equator and prime meridian, respectively). Thus, a
coordinate of the Earth is a value of type $\type{Number}^2$. The coordinate at the equator and prime meridian we will
label as "the origin" and arbitrarily say is \coord{0,0}.



$$\begin{tikzcd}
A \arrow[swap]{dr}{f} \cdr{i_1} & A + B \arrow[dashed]{d}{s} & \cdl{i_2} \arrow{dl}{g} B \\
\phantom{} & X &
\end{tikzcd}$$\change{this shouldn't be here}

\section{Vectors}
\section{Bases}
In our map example, the two coordinates we used to describe the space were latitude and longitude--more precisely, a
vector pointing along the north-south axis, \coord{0,1}, and one pointing along the west-east axis, \coord{1,0}.
However, this is simply a matter of convention, there is no reason we couldn't have used one along the
northwest-southeast axis, \coord{1,1}, and another along northeast-southwest, \coord{1,-1}. In fact, we could even use
one pointing north, \coord{0,1} and one pointing north-north-east, \coord{$\frac{1}{4}$,1}. As it turns out, for our
map example, any two vectors will satisfy\footnote{Conceptually, at least. Actually \textit{using} these systems would
be exceptionally frustrating, though possible in principle,} us as a coordinate-system, so long as they are not both
pointing in the same direction.  It does us no good to have two coordinates which are both north-south, because we wold
never be able to describe what "east" was.\todo{we need to talk about vectors before this}

The arbitrary vectors we choose as our coordinates form what is known as a \textit{basis}.

\defn{basis}{A basis of a space is a set of vectors which act as the coordinate system. A choice of basis is completely
arbitrary, though different bases will make salient different features of the space, as we will see later. In general,
an $n$-dimensional space requires a basis consisting of $n$ vectors.}

While the north and west vectors are arbitrary choices for the basis of our map, it turns out there is a good reason for
why we might choose them, which we will discuss later in the section on projections\todo{get the reference for
Spaces::Projections}.

\subsection{Color in RGB-space}
While the choice of a basis is arbitrary, different bases can show interesting features in the underlying space. To
drive home this point\improvement{diction, please.}, let us consider \textit{color-space}--the space of colors. As you
might imagine, in color-space, we can use coordinates to describe different colors. In fact, this is how computers
represent colors, so we can rest assured that we're not just making things up.

If you recall your early childhood education, there are three primary colors--colors which can't be made up of other
colors. In contrast, all other colors can be made by mixing the primary colors together in different proportions. This
should set off the mental bells in your mind: this is a perfect example of a space! In fact, our basis can simply be
these primary colors, with coordinates being a percentage between 0\% and 100\%, with 0\% meaning there is none of the
color present, and 100\% being as much as possible. In the additive color model (how lights form colors, rather than how
paint forms color), the three primary colors are \value{red}, \value{green} and \value{blue} (or RGB for short). Thus,
we shall describe our basis to be the three vectors:

\begin{align*}
    \value{red:}   & \qquad \coord{100,0,0} \\
    \value{green:} & \qquad \coord{0,100,0} \\
    \value{blue:}  & \qquad \coord{0,0,100}
\end{align*}

\fig{spaces/rgb.png}{An illustration of RGB-space.\todo{find a better image or something maybe even cite this yolo
swag}}{0.25}

You'll notice that colors which look similar to one another are close to one another in RGB-space. This is a phenomenon
known as clustering--similar colors have nearby coordinates. We will study clustering in more detail later.


\subsection{Color in HSV-space}

\section{Projections}

\section{Clustering}



\section{Searching}






\chapter{Algorithms: The Language of Decision}
\chapter{Monads: The Language of Progression}
\chapter{Isomorphisms: The Language of Identity}
\chapter{Asymptotes: The Language of Refinement}


\end{document}

